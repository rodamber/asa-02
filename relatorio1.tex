\documentclass[12pt, a4paper]{article}

\usepackage[T1]{fontenc}
\usepackage[utf8]{inputenc}
\usepackage[portuguese]{babel}
\usepackage{authblk}
\usepackage{enumitem}

\setlist[description]{labelindent=1cm}

\title{\textbf{Relatório do 1º projecto de ASA}}
\author{Rodrigo André Moreira Bernardo \\ ist178942}
\affil{Instituto Superior Técnico}

\date{\today}


\begin{document}
\maketitle

\section{O Problema}

\subsection{Introdução}
\paragraph{}
Paul Erdős, reconhecido matemático do século XX, colaborou com mais de 500
matemáticos na co-autoria de artigos científicos.
O número de Erdős é definido como a distância colaborativa de uma pessoa a Paul
Erdős. Paul Erdős tem número 0, um co-autor de artigos com Paul Erdős tem número
de Erdős 1, um co-autor de artigos com co-autores de Paul Erdős tem número 2 e
assim por adiante [1].

\subsection{Objectivo}
\paragraph{}
Dado um input que identifique Erdős, um conjunto de colaboradores seus e as
colaborações entre si, determinar os números de Erdős de cada um dos
colaboradores. O output deverá indicar o maior valor, \textit{M}, de número de Erdős
identificado, assim como informação do número de pessoas com número de
Erdős \textit{i}, com 1 $\leq$ \textit{i} $\leq$ \textit{M}.

\section{A Solução}
\paragraph{}
O programa foi implementado em linguagem C.
A solução passa por executar uma procura em largura primeiro (BFS) sobre um
grafo cujos vértices representam Erdős e os colaboradores, e cujos arcos
representam as colaborações. Com a BFS conseguimos obter a distância
colaborativa entre cada colaborador e Erdős. Por fim, apenas é necessário
efectuar duas passagens pelos colaboradores: uma para determinar \textit{M} e
outra para determinar os valores \textit{i} (ver \textit{Objectivo}).

\subsection{A Representação}
\paragraph{}
Tanto Paul Erdős comos os colaboradores são representados por um inteiro.
A representação do grafo é em listas de adjacências. Mais pormenorizadamente,
este é representado através de um vector de listas ligadas simples, com tamanho
igual ao número de  vértices. Cada lista tem a informação relativa ao vértice
correspondente, utilizada no algoritmo BFS (cor, distância e predecessor), assim
como um apontador para o primeiro elemento da lista. Cada elemento da lista
contém um inteiro representativo do colaborador e um apontador para o próximo
elemento.

\subsection{O Algoritmo}
\paragraph{}
O algoritmo BFS utilizado é um adaptado da página 595 da terceira edição do
livro \textit{Introduction to Algorithms} [2].


\section{Análise Teórica}
\subsection{Avaliação}
\paragraph{}
Do programa destacam-se os seguintes blocos:

\begin{description}
    \item[i.] A inicialização do vector de listas de adjacências, nas linhas
        211-214;
    \item[ii.] A inserção dos arcos no vector, nas linhas 217-233;
    \item[iii.] A BFS, nas linhas 239 e 150-189;
    \item[iv.] A determinação de \textit{M} (ver \textit{Objectivo}), nas linhas
        242-247;
    \item[v.] A determinação dos valores \textit{i} (ver \textit{Objectivo}), nas linhas
        254-258;
    \item[vi.] A impressão do output, nas linhas 261-265;
    \item[vii.] As operações de alocação e libertação de memória.
\end{description}

Sejam \textit{V} o número de vértices e \textit{E} o número de arcos do grafo.
A complexidade das operações realizadas nos pontos \textbf{i.}, \textbf{iv.},
\textbf{v.} é \textit{O(V)} (ciclos \textit{for} no tamanho do número de vértices).
Em relação ao ponto \textbf{ii.}, deve ser referido que a operação de inserção
em lista é \textit{O(1)}, pois cada novo elemento é inserido no início da lista.
Assim, como a operação \textit{scanf} também é constante, concluí-se que o tempo
de execução deste ponto é \textit{O(E)}, visto que é um ciclo \textit{for} no
tamanho do número de arcos.
O tempo de execução do algoritmo BFS é analisado em [2], na página 597, sendo
este \textit{O(V + E)} (ponto \textbf{iii.}). De facto, o tempo gasto em
operações relacionadas com a \textit{queue} é \textit{O(V)}, assim como na
inicialização. Para além disso, cada lista de adjacências é passada apenas uma
vez, sendo que o tempo para passar cada uma é \textit{O(E)}.
Como \textit{M} $\leq$ \textit{E}, sai que a complexidade relativa ao
ponto \textbf{vi.} é \textit{O(E)}.
Por fim, relativamente ao ponto \textbf{vii.}, alocou-se memória para o vector
de listas de adjacências, para as listas em si, para os seus elementos e para a
\textit{queue} utilizada no algoritmo BFS. Tem-se que o número de listas é igual
a \textit{V}, que o número de elementos que cada uma tem é \textit{O(E)} e que o
tamanho da \textit{queue} é \textit{O(V)}.

\subsection{Conclusões}
\paragraph{}
Conclui-se que o tempo de execução do programa é \textit{O(V + E)}.
A memória ocupada é, também, \textit{O(V + E)}.


\section{Avaliação Experimental}
\paragraph{}
Após correr alguns testes, foram obtidos os seguintes tempos (totais) de execução com a
ferramenta \textit{Unix time}.

\begin{center}
\begin{tabular}{| r | r | r |}
    \hline
    Vértices & Arcos & Tempo (s)\\
    \hline
    500 & 5 000 & 0,005 \\
    \hline
    15 000 & 60 000& 0,063 \\
    \hline
    30 000 & 60 000 & 0,064 \\
    \hline
    30 000 & 90 000 & 0,080 \\
    \hline
    30 000 & 120 000 & 0,105 \\
    \hline
    45 000 & 60 000 & 0,067 \\
    \hline
    45 000 & 90 000 & 0,088 \\
    \hline
    45 000 & 120 000 & 0,109 \\
    \hline
    45 000 & 150 000 & 0,133 \\
    \hline
\end{tabular}
\end{center}

\paragraph{}
Também foi obtida a memória utilizada, com recurso à ferramenta
\textit{valgrind}.

\begin{center}
\begin{tabular}{| r | r | r |}
    \hline
    Vértices & Arcos & Memória (bytes)\\
    \hline
    15 000 & 60 000 & 2 637 892 \\
    \hline
    30 000 & 60 000 & 3 360 036 \\
    \hline
    30 000 & 90 000 & 4 320 036 \\
    \hline
    30 000 & 120 000 & 5 280 036 \\
    \hline
    45 000 & 60 000 & 4 080 036 \\
    \hline
    45 000 & 90 000 & 5 040 036 \\
    \hline
    45 000 & 120 000 & 6 000 036 \\
    \hline
    45 000 & 150 000 & 6 960 036 \\
    \hline
\end{tabular}
\end{center}

\paragraph{}
Com recurso às tabelas, é possível identificar que o programa corre linearmente
no tamanho do input, como seria de esperar. Destaca-se, no entanto, que o
programa nota uma sobrecarga maior se aumentarmos os arcos do que se aumentarmos
os vértices na mesma quantidade. Isto verifica-se porque no algoritmo BFS
existem mais passagem sobre os arcos do que sobre os vértices, no caso do tempo.
O impacto na memória do aumento dos arcos já seria de esperar, obviamente.

\section{Referências}
\paragraph{}
\indent[1] Enunciado do primeiro projecto de ASA.

[2] Cormen, Thomas H.; Leiserson, Charles E.; Rivest, Ronald. L; Stein, Clifford
(2009). \textit{Introduction to Algorithms} (3rd ed.). MIT Press.

\end{document}
