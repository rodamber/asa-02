\documentclass[12pt, a4paper]{article}

\usepackage[T1]{fontenc}
\usepackage[utf8]{inputenc}
\usepackage[portuguese]{babel}
\usepackage{authblk}
\usepackage{enumitem}

\setlist[description]{labelindent=1cm}

\title{\textbf{Relatório do 1º projecto de ASA}}
\author{Rodrigo André Moreira Bernardo \\ ist178942}
\affil{Instituto Superior Técnico}

\date{\today}


\begin{document}
\maketitle

\section{O Problema}

\subsection{Introdução}
\paragraph{}
A empresa Coelho e Caracol, Lda. faz transporte de mercadorias. O Sr. Coelho
trata do transporte, enquanto o Sr. Caracol fica no escritório a fazer o
planeamento das rotas, sendo que cada uma consiste numa sequência de
localidades. Cada conexão entre localidades tem a si associada um valor de
perda, que resulta de subtrair a receita ao custo.

\subsection{Objectivo}
\paragraph{}
Dado um input que identifique a sede da empresa, o número de localidades e as
conexões entre si, assim como o valor de perda de cada uma, determinar a perda
mínima desde a sede até cada localidade. O output deverá indicar para cada
localidade o respectivo valor de perda. No entanto, caso uma localidade seja
impossível de alcançar, então o seu valor de perda deve ser "U" e, caso seja
impossível definir essa perda, então dizemos que o seu valor de perda é "I".

\section{A Solução}
\paragraph{}
O programa foi implementado em linguagem C.
A solução passa por executar o algoritmo de Bellman-Ford para caminhos mais
curtos sobre um grafo cujos nós representam as localidades e cujos arcos
representam as conexões entre as localidades.

\subsection{A Representação}
\paragraph{}

\subsection{O Algoritmo}
\paragraph{}

\section{Análise Teórica}
\subsection{Avaliação}
\paragraph{}

\section{Análise Experimental}
\paragraph{}

\section{Referências}
\paragraph{}
\indent[1] Enunciado do primeiro projecto de ASA.

\end{document}
