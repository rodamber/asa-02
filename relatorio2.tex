\documentclass[12pt, a4paper]{article}

\usepackage[T1]{fontenc}
\usepackage[utf8]{inputenc}
\usepackage[portuguese]{babel}
\usepackage{authblk}
\usepackage{enumitem}
\usepackage{graphicx}
\usepackage{tikz}
\usepackage{pgfplots}

\graphicspath{{./plots/}}
\pgfplotsset{compat=1.10,width=10cm}

\setlist[description]{labelindent=1cm}

\title{\textbf{Relatório do 2º projecto de ASA}}
\author{Rodrigo André Moreira Bernardo \\ ist178942}
\affil{Instituto Superior Técnico}

\date{\today}


\begin{document}
\maketitle

\section{O Problema}

\subsection{Introdução}
\paragraph{}
A empresa Coelho e Caracol, Lda.\ faz transporte de mercadorias. O Sr. Coelho
trata do transporte, enquanto o Sr. Caracol fica no escritório a fazer o
planeamento das rotas, sendo que cada uma consiste numa sequência de
localidades. Cada conexão entre localidades tem a si associada um valor de
perda, que resulta de subtrair a receita ao custo.

\subsection{Objectivo}
\paragraph{}
Dado um input que identifique a sede da empresa, o número de localidades e as
conexões entre si, assim como o valor de perda de cada uma, determinar a perda
mínima desde a sede até cada localidade. O output deverá indicar para cada
localidade o respectivo valor de perda. No entanto, caso uma localidade seja
impossível de alcançar, então o seu valor de perda deve ser "U"\ e, caso seja
impossível definir essa perda, então dizemos que o seu valor de perda é "I".

\section{A Solução}
\paragraph{}
O programa foi implementado em linguagem C.  A solução passa por executar o
algoritmo de Bellman-Ford para caminhos mais curtos sobre um grafo, \textit{G},
cujos vértices representam as localidades e cujas arestas representam as
conexões entre as localidades. No último ciclo do algoritmo, caso se verifique
que ainda seria possível relaxar uma determinada aresta (\textit{u},
\textit{v}), ou seja, se existir ciclos negativos, então é feita uma procura em
largura primeiro (BFS) com início em \textit{v}, marcando a estimativa de
caminho mais curto de cada vértice visitado a $-\infty$, o que significa que é
um vértice "I".

\subsection{A Representação}
\paragraph{}
Cada localidade é representada por um inteiro. A representação do grafo é em
listas de adjacências. Mais pormenorizadamente, um grafo é uma estrutura
composta por um vector de ponteiros para vértices e por um inteiro que indica o
seu número de vértices (linhas 57-60). Cada vértice é uma estrutura com vários
elementos, entre eles uma chave que o representa e um ponteiro para uma aresta
adjacente (linhas 9-20). Cada aresta depois tem um ponteiro para a aresta
adjacente seguinte (linhas 26-34). Propriedades características de cada
algoritmo (como estimativa de caminho mais curto no algoritmo de Bellman-Ford ou
cor na BFS) estão nas estruturas dos vértices e das arestas (em vez de existirem
vectores para cada propriedade) para ser mais fácil aceder a essas propriedades
nas várias partes do programa.
\paragraph{}
Para representar $-\infty$ e $+\infty$, foram utilizados INT\_MIN e INT\_MAX,
respectivamente. Para acomodar esta escolha, foi criada uma função de soma
(linhas 254-258), de forma a evitar \textit{overflows}.

\subsection{O Algoritmo}
\paragraph{}
O algoritmo é uma versão modificada do algoritmo de Bellman-Ford. No entanto,
foram feitas duas optimizações. A primeira passa por terminar o algoritmo caso
tenha havido uma iteração onde não se verificasse relaxamento de arestas. A
segunda passa por, a cada iteração, relaxar apenas arestas cujo vértice da qual
partem tenho visto a sua estimativa de caminho mais curto alterada desde a
última vez que essas arestas foram relaxadas.
\paragraph{}
Em relação à detecção dos vértices "I", é também verificado se a estimativa de
caminho mais curto de cada vértice encontrado não é já $-\infty$.  Se for,
termina-se a BFS (ou nem sequer se inicia), de forma a diminuir o número de
pesquisas redundantes.

\section{Análise Teórica}
\paragraph{}
Sejam \textit{V} e \textit{E} o número de vértices e arestas do grafo,
respectivamente.  O programa começa por ler as duas primeiras linhas do input,
obtendo \textit{N}, o número de localidades, \textit{C}, o número de custos, e
\textit{H}, um inteiro que representa a sede da empresa. Depois inicializa o
grafo com \textit{V=N} vértices, que demora O(V). Ler as linhas
restantes para adicionar as arestas é O(E), tendo em conta que
adicionar uma nova aresta ao grafo é O(1), \textit{E=C}, e considerando
que as operações de ler linhas e alocar memória são O(1). Até agora
temos uma complexidade para o tempo de O(V + E).

\paragraph{}
O algoritmo (modificado) de Bellman-Ford começa por inicializar todos os
vértices, que é O(V). A seguir tem um ciclo com, no máximo,
V-1 iterações, onde se aplica a operação de relaxar as arestas. No
máximo, todos as arestas são relaxadas. Como relaxar um arco é O(1),
temos que o ciclo é O(VE). Depois, se tiverem ocorrido relaxamentos na
iteração V-1, o algoritmo verifica se existem ciclos negativos. O
algoritmo de Bellman-Ford garante que após V-1 iterações de
relaxamentos todos os vértices têm as suas estimativas de caminho mais curto
correctas, a não ser que existam ciclos negativos. Nesta versão, quando se
detectam ciclos negativos é feita uma BFS para atingir todos os vértices que o
ciclo "atinge"\ e alterar a sua estimativa de caminho mais curto para $-\infty$.
É feita apenas uma BFS para cada ciclo negativo que existir.  Concluí-se que o
conjunto de todas BFS tem complexidade temporal O(V + E).

\paragraph{}
Imprimir o output é simples de analisar. Tem-se um ciclo para percorrer todos os
vértices. Considerando as operações de impressão como sendo O(1),
imprimir todo o output é O(V).

\paragraph{}
No final tem-se O(V + E) + O(VE) + O(V) = O(VE) para o tempo.

\paragraph{}
Em relação à complexidade espacial, excepto algumas alocações independentes do
tamanho do input (que são, portanto, O(1)), apenas se fazem alocações a
estruturas de dados com tamanho linear em \textit{V} e em \textit{E}, pelo que a
complexidade espacial é O(V + E).

\section{Análise Experimental}
\paragraph{}
Com recurso ao gerador fornecido, o \textit{p2-gen}, foram criados 500 grafos,
com um número de vértices entre 10 e 49910, sendo que, para cada grafo, se tem
E=2V. Foram obtidos os tempos totais de execução com a ferramenta \textit{time}
da \textit{BASH} e a memória ocupada com a ferramenta \textit{valgrind}.
\paragraph{}
Verifica-se que a memória ocupada é linear sobre V+E.
Já em relação ao tempo de execução, o gráfico já não se assemelha tanto a uma
linha, como se poderia esperar. No entanto, ao contrário da memória ocupada, o
tempo de execução depende de factores externos, como a sobrecarga do computador.

\paragraph{}

\centering
\begin{tikzpicture}[trim axis left, trim axis right]
    \begin{axis}[
        height=7cm,
        width=0.8\textwidth,
        enlargelimits=false,
        title={Tempo de execução em função de VE},
        xlabel={VE},
        ylabel={Tempo de execução (s)},
        scaled y ticks = false,
        yticklabel style={/pgf/number format/fixed},
        ymajorgrids=true,
        grid style=dashed,
    ]
        \addplot+[
            only marks,
            scatter,
            mark=x,
            mark size=3pt,
        ]
        table{tests/data/times.dat};
    \end{axis}
\end{tikzpicture}

\paragraph{}

\centering
\begin{tikzpicture}[trim axis left, trim axis right]
    \begin{axis}[
        height=7cm,
        width=0.8\textwidth,
        enlargelimits=false,
        title={Memória ocupada em função de V + E},
        xlabel={V+E},
        scaled x ticks = base 10:-4,
        every y tick scale label/.style={at={(-0.060,1)}},
        ylabel={Memória ocupada (bytes)},
        ymajorgrids=true,
        grid style=dashed,
    ]
        \addplot+[
            only marks,
            scatter,
            mark=x,
            mark size=3pt,
        ]
        table{tests/data/mems.dat};
    \end{axis}
\end{tikzpicture}

\end{document}

















